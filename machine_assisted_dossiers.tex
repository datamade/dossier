\documentclass[format=siggraph, review=true]{acmart}
\acmConference{Computation + Journalism Symposium}{October 2017}{Chicago, IL}
\title{Machine Assisted Dossiers}
\author{Forest Gregg}
\affiliation{DataMade}
\email{fgregg@datamade.us}
\author{Jean Cochrane}
\affiliation{DataMade}
\email{jean.cochrane@datamade.us}
\author{Timothy McGovern}
\affiliation{O'Reilly Media}
\email{timmymcg@gmail.com}
\begin{document}


\begin{abstract}One of the great disappointments of big data is that so much
  of it is bad data. It is unreliable, ambiguous, and
  contradictory. Developing an accurate image of the world still
  requires discernment, sorting, and judgment.

  We are still are only beginning to building technologies that are
  complementary to these human capacities--allowing for
  scale.  In this paper, we present the capabilities we believe an
  adequate knowledge system must have, drawing heavily from the field
  of genealogy and our own work modeling international security forces
  and campaign finance.

  We'll discuss the overall requirements for such a system and try to envision its user experience and its data architecture; we'll also survey where currently available technologies can fill in the gaps between the two. 
\end{abstract}

\maketitle

Most of the time, investigative journalists use data and documents
that were not made for them. The material that they FOIA, scrape, get
leaked, download from data portals, or dig up from the archives
were not made for journalists or intended to help answer the
questions she is reporting.

In order to do their work, journalists have to struggle to get access
to documents or administrative data; manage large collections of
source files; extract the relevant information; identify ambiguous
references; and reconcile conflicting claims.

That journalists accomplish these tasks and relatively quickly is a
testament to their skills as researchers. These skills though are
private and focused on the investigations at hand. The work does not
accumulate a store of knowledge that can be reused by future
journalists for new investigations.

The promise of computers, meantime, is to offer speed (enabling the analysis of larger corpora of source material), scale (enabling the analysis of larger webs of interrelated facts), and memory (enabling knowledge to be stored and shared, whether for reproducing an analysis or for bringing the knowledge to bear on a new problem). 

Though journalists and newsrooms would benefit from building shared
dossiers of the key people and organizations in a beat, this is not a
common practice. The reasons for this are various, but we believe a
substantial barrier is that existing tools do not provide immediate
benefits to a journalist in the middle of an investigation. It may be
very helpful for a journalist to look up what the news room already
knows about a city councilor in an internal wiki, but adding new
content to the wiki is just an additional chore. 

At DataMade, we have been building and researching knowledge
management systems that can help the investigators in the many stages of
research and which, almost as a byproduct, produce shared dossiers of
people and organizations. In this paper, we discuss the capabilities
that a well designed system should possess, from the perspective of information architecture and end-user experience.

\section{Existing Tools}
Existing knowledge management systems prioritize flexibility for the
individual researcher over general utility for a team, or for a wider
audience. When individual researchers are given great freedom over the
structure and organization of the information they collect, other
stakeholders lose out, since the knowledge must be translated into a form
they can understand. When the needs of a wider audience are prioritized,
however, the system loses capacity for tolerating ambiguity, and the
liberty of individual researchers is constrained.

Here, we consider the affordances and limitations of a few of the most
popular tools for knowledge management in research contexts.

\begin{description}
  \item [Notecards] Collecting and organizing analog files remains
    a popular research method, for good reason. Using notecards, an
    individual researcher is free to organize and reorganize a
    knowledge system however she pleases. Notecards neatly illustrate
    the tradeoff between flexibility and utility involved in designing
    a knowledge system: they offer total control to the individual
    researcher at great cost to the general public, for whom the system
    must be painstakingly translated to be of any use at all.

  \item [Scrivener] Composition tools like Scrivener port the visual
    and mechanical metaphors of analog research methods to digital
    contexts. With these tools, information can be recorded in digitized
    form, but it is still completely unstructured. The individual
    researcher is able to keep track of more and more different kinds
    of information, but translation is still required in order for others
    to make sense of it.

  \item [Wikis] In contrast to notecards and digital composition tools,
    wikis prioritize the needs of a collective of stakeholders over the
    individual researcher. Information is sometimes structured and
    attached to specific sources, and the process that produced the
    knowledge can be preserved in discussion forums. In a wiki, however,
    individual researchers lose a great deal of power in order to create
    value for the collective: knowledge is subject to collective approval,
    and conflicting truth claims must be reconciled fully and immediately.
\end{description}

\section{Scope Conditions}
In order to provide more support for journalists and other
researchers, a system must be designed for a defined universe of
knowledge to manage --- the types of organizations, persons, and
events; the attributes of those entities; and the relations amongst
them.

With these set, the designer of the system can identify source
material with potential relevance, what pieces of information that
will be important in the source material, what facets of the
information will be useful to index, and what types of claims are
congruent or incompatible.


The more defined the field of knowledge, the more that information
technology can aid the production of that knowledge. However, given
the current costs of building, a limited field of knowledge is not
sufficient. These types of systems should only be
built where three conditions are met. First, there is fairly narrow
knowledge area that has wide and durable interest. Second, the number
concrete instances of knowledge is much larger than one person can
manage using private skills. Three, there is a large corpus of primary
source material that can be used to develop concrete knowledge in a
repeatable and separable manner. 

Some examples include:

\begin{description}
  \item [Geneology] Who were the parents of whom. When and where was a person
    born and when and where did they die.
  \item [Corporate Beneficiaries] Who ultimately owns or controls an
    organization, which may be owned by a chain of shell organizations
  \item [Security Forces] What is the organizational structure of
    security forces. Who are the commanding officers of units and
    what has been their careers.
  \item [Campaign Finance] Who, ultimately, gave money to which
    political campaigns, even through intermediaries.
  \item [Human Rights Violations] Who and how many people have been
    killed in an armed conflict
  \item [Customer Resource Management] The nature and relationships of individuals and organizations; the history of contacts between them. 
\end{description}

Consider three different organizations, each working in one of these problem
domains, each with separate missions but similar sets of problems:

Security Force Monitor is a nonprofit based out of Columbia University
with the mission of tracking of security force activity in conflict areas
around the world. Maintaining a robust knowledge system is a key part of
their mission, but it is a difficult task: keeping track of security forces
requires keeping track of large quantities of ambiguous
information—information that has been retrieved from unreliable sources,
primarily secondary news reports, and that is then adjudicated by staff
members who lack authoritative context. For any given assertion about a
human rights violation to be credible, the knowledge system must know
where each piece of information came from, which researcher catalogued it,
and the degree of confidence the researcher had in the assertion. Yet for
the system to be useful to an audience outside of the internal research
team, it must also be capable of adjudicating conflicting claims and
presenting a structured view of the security forces and the incidents
they have been involved in.

Invisible Institute is a nonprofit media organization in Chicago that
reports on misconduct in the Chicago Police Department (CPD). Following
their legal victory in Kalven v. City of Chicago in 2014, which opened
up over 56,000 complaints against CPD officers to the public, they have
released a database of complaint records to the public with detailed
demographic information about complainants and officers involved in
alleged misconduct. The organization would like to be able to produce
a dossier of CPD officers in order to to link these complaints to other
kinds of misconduct records, but the source records that they FOIA from
the CPD and the City of Chicago are typically difficult to parse, and
lack the unique identifiers that would allow them to unambiguously
assign responsibility for incidents.

FamilySearch is a service provided by the Church of Latter-Day Saints
that seeks to build detailed genealogies using demographic records.
The service sources much of its information from old census records,
which are OCRed and then interpreted by staff members, but it also allows
users to upload their own records and contest genealogies recorded by the
system. Every claim must be tied to a specific record, and contested
assertions are eventually adjudicated.

As we describe the necessary features of machine-assisted dossiers, we will
refer back to the problems and solutions that these three organizations
engage with in their attempts to build comprehensive knowledge systems.

\section{Document Management}
The system must have the capability to collect the source material
which will be the evidence to support the development of
knowledge and make those materials convenient for the purposes of
research. The set of problems here are largely covered under the field
of document management and there already exists many excellent tools
for this portion of the task.

For our purposes, we are using ``document'' to mean any type of source
material. They are most often different types of files: word
processing documents, PDFs, markup, spreadsheets, etc. They could
include audio testimony, news articles, or FOIAed documents.

Beyond storage, the three key capabilities for document management
portion of the system is to capture the provenance of source material;
converting the source material into convenient formats; and indexing
the documents in support of research.

\subsection{Provenance}
As the knowledge developed within the system ultimately depends upon
source documents, the provenance of those documents must be
recorded. Who or what (if it was an automated scraper) collected the
material, when, from what original location. The original forms of the
documents must be preserved.

\subsection{Formatting}
Often, source material is not in a convenient format for computer
processing. A file may be in an awkward or proprietary format, or a
document may only be collected as scanned image. As part of the
research process, the material may be converted to a form that allows
for easier processing. Sometimes this conversion is unproblematic,
like for many file format conversions. Sometimes, the conversion is
very error prone such as OCRing a scanned document or human
transcription of audio recordings.

Regardless, the details about the conversion need to be recorded ---
who or what did the conversion and when. If the process was done by
computer, steps must be taken to ensure that the conversion is
completely reproducible. As technologies or other capabilities
improve, the journalist or researcher may want to reconvert existing
documents and conversion metadata supports this.

\subsection{Indexing}
Finally, appropriately formatted documents should be indexed for the
next stage of research. While the systems should to full-text indexing
to allow for flexible searches, the system should also attempt to
index the documents on facets relevant to the target knowledge
area. This means that the system should attempt to identify references
within a document to the types of person, organizations, places, and
events that the overall system is concerned with.

If the source material is already highly structured, this can be
simple. However, if the material is free text, then the system should
be attempt to identify references using Named Entity Recognition
techniques. 

Indexing is the most basic form of computer analysis of documents. It enables non-trivial analysis of topics and relevant entities through simple counting and co-location analysis, and when combined with minimal provenance data (chronology), can provide evidence of change over time. Indexing also provides the basis for building a databases of named entities.

\section{Entity Management}

Once we have secured our documents, regularized them, and indexed them, producing a searchable list of entities is rewarding both immediately and in the long term. ``What do we know about Jane Tye?'' is a question that can be usefully answered with a keyword-in-context (KWIC) search, even when the search produces hundreds of hits. Entity management also can entail using machine learning techniques to preliminarily classify entities. This may entail sifting out individuals from organizations, or well-connected individuals from peripheral ones, but we can start to see the basis for machine-assisted analysis of large data sets.

\section{Claim Management}
Once the documentary base is prepared, the work of extracting claims
about the world from those documents, resolving those claims to
reference particular entities, and reconciling conflicting claims can
begin. Unlike document management, the practices for what we call
``claim management'' are still developing.

\subsection{Extracting Claims}
Given a source document, a journalist will extract claims relevant to
the entity of interest. If they are researching campaign finance, they
might be interested in the extracting the claim that ``John Smith''
gave \$500 to ``Citizens for Better Citizens'' on December 11, 2017
from a financial disclosure form of the ``Citizens for Better
Citizens'' political action committee.

While system should allow the journalist to extract the claims in the
most natural, practicable manner, the system should decompose compound
claims into simpler claims. For example, the above
claim could be broken down as follows:

\begin{itemize}
\item ``John Smith'' made a contribution to this committee
\item ``John Smith'' made a contribution during the reporting period of this disclosure
\item ``John Smith'' made a contribution to this committee on December 11, 2017
\item ``John Smith'' gave this committee \$500
\item somebody gave this committee \$500 during this reporting period
\item somebody gave this committee \$500 on December 11, 2017
\end{itemize}

Extracting claims is effortful. While full compound claims are often
incorrect in some particular, elements of the claim can often be
maintained and this decomposition preserves some of the initial work.

The types of claims that can be recorded are those that the system has
been designed to handle. The system must capture and preserve data on
who or what extracted the claim and when this extraction occurred.

\subsection{Resolving Claims}
In the cases we deal with, there is almost always ambiguity about
which particular entity a claim in a document is about. While a
journalist will believe they are extracting a claim about a particular
person or organization, they can find that they have been
mistaken. Using the above example, a journalist can mistakenly
attribute a campaign contribution to the wrong ``John Smith.''

The knowledge system must allow for this type of ambiguity by avoiding
modeling extracted claims as claims about particular
instances. Internally, an extracted claim attached to a particular
dossier would be modeled as two related claims. The first is the one
extracted from the document: `A person with the name ``John Smith''
gave \$500 to this committee.' The second claim is `The person who is
referenced in the extracted claim is the person who this system
uniquely indexes with the unique identifier ``1313515'''

If claims about particularly people are split in this way, then
extracted claims can be re-assigned to the correct entities as the
journalist develops a more accurate picture.


\subsection{Reconciling Claims}
Since the knowledge systems work within limited fields of knowledge,
the system designers can elaborate a model of how the this portion of
the world should operate. This can allow for the flagging of claims
that are logically incompatible. For example, campaign committees have
founding dates, so there should be no contributions from a campaign
committee before it was founded. 

With or without the help of system, the journalist must decide which
claims are compatible and decide which, if any, they want to
accept. The system must be able to record the journalist's belief about
the validity of a claim about a particular person or
organization. These decisions should be reversible.

\subsection{Uncertainty in Claims}
Since the work of Lotfi Zadeh [TODO: Citation!], the worlds of logic and computer science have reckoned with uncertainty and partial truth using computation-friendly tools. Incorporating some measure of uncertainty into any calims (and enabling the adjustment of this level of uncertainty as new information is added) is an important part of claims management as well as detecting and reconciling conflicting claims. As Charles Ragin put it, ``fuzzy sets have the potential to transform research that is oriented toward `discovery,' toward gaining new insights about the world.'' (2000) Fuzzy logic enables us to test different understandings against the evidence. 

Recording (and modifying) the reporter's uncertainty about a claim only records half of the uncertainty that is possible in the reportage/world-being-reported-on system. In addition to the uncertainty in knowledge, which can be modeled, there is uncertainty in the world—which can be exploited for political aims outside the law. Giorgio Agamben [TODO: Citation] and others have explored the implications of these ``states of exception'', where ``public law meets political necessity,'' that is to say, where the normal course of law is suspended in order to deal with extraordinary political phenomena.  These are often sites not merely for temporary resolution of drastic problems, but for radical shifts in power arrangements, or for activities outside the realm of the thinkable.  

A paradigmatic site of exception is the American military base at Guantànamo Bay; the multiple conflicting claims about political sovereignty at Guantànamo have enabled military and civilian commanders to exercise authority outside the oversight of courts—enabled by presidential order that was intended to be a temporary response to the attacks of September 11. 

As Agamben and others have pointed out, states of exception have both potential positive (the German Constitution, for example, explicitly allows for an extralegal right of resistance to attempts to legally abolish the constitutional order) and negative consequences, but in every case, they are a situation fraught with the danger of allowing lawlessness to take hold. Because of their political importance, early detection of states of exception, where formal structures of social order (sovereignty, chain of command, etc.) are breaking down or are blurred, is an important area where machine-assisted claims analysis could be of importance. 

\section{Differences in Implementation}

Different knowledge domains have different research needs. When
implementing a machine-assisted dossier, trade-offs will have to be made
between the flexibility afforded to individual researchers and the
usability of the system for a wider collective. Specific implementations
of the system we propose will vary in the degree to which they permit
the following features:

\begin{description}
  \item [Conflict resolution] It may be beneficial to some systems to
    allow conflicting claims about entities or attributes to coexist,
    and to expose these conflicts to users. Other systems will want
    to enforce a unitary vision of the world.

  \item [Custom attachments] Researchers may wish to collect unstructured
    data in the system, in order to keep track of information they might
    need at a later date, or to pursue promising lines of inquiry that
    have not yet proven to be valuable to the collective.

  \item [Interface between document collection and claim management] When
    conflicting claims get resolved, it is likely that logic will
    want to propagate down toward decisions made in the lower levels
    of document collection. Some systems will want to adjust these lower
    levels automatically; others will want to expose inconsistencies to
    researchers through the system interface, to allow them to adjudicate
    the claims. Still others may wish to permit logical inconsistencies.
\end{description}

\section{The Path from Document Management to Claim Management}
How do we get there from here? What parts of a claim management system already exist (or can be created using existing technology)? What parts of a claim management system can be modularized or extended? What is generalizable about different types of claim management and warrant management? 

\subsection{Minimum Viable Product: Manage Easily Disproved Claims}
Many of the problems of claims extraction and mangement are already known problems to software creators, that is to say, they already defined in ML terms: 
- Entity extraction (including attribution)
- Geo-location
- Clustering/similarity
- Topic Modeling
- Social Network Analysis
- Sentiment Analysis

These may be appealing as the kinds of claims we want to start working with, but only some of these are immediately amenable to a claims management system. While it may be possible to cobble together a system of APIs and analyses, it seems to us that a better approach is to start from the the types of claims that are most amenable to being formalized and tracked in database form. 

Where therefore propose beginning not with the most well-studied problems in commercial computing applications (though we'll aim to make use of these) but rather with the most easily disproven claims. 

This means starting with existence (entity extraction) and location for natural persons. Then moves onto formally defined social entities (corporations, military units) and formally defined relations between entities (ownership structures, chain of command, etc.). As a note of caution, while the tools of entity extraction and geo-location map well onto disprovable claims, many of the other tools of data science do not. While clustering, topic modeling, and social network analysis may point to the existence of relationships among entities, they do not distinguish between formal and informal networks and similarities. For the purposes of recording, analyzing, and perhaps most crucially, sharing claims, these methods don't allow for a generalizable method of looking at claims for a different purpose.

\section{Examples of systems for managing structured claims}
Let's look briefly at two more systems for generalizably managing structured claims. The first is the Zooniverse project, which enables crowdsourcing of documentary analysis in many fields, and the other is the well-known CRM, Salesforce. 

Zooniverse makes it possible to crowdsource the creation of structured data from unstructured or handwritten data. [TODO: insert screenshots. https://www.zooniverse.org/help  or ] It's been developed for and used by researchers in many different fields, from astronomy to history. The general framework for developing a project is as follows.
- Experts create the (known) structures for the information. 
- Raw evidence is parsed (in this case by humans) and recorded (with a chain of assertion). 
- Conflicting warrants are raised as exceptions (delivered to more human interpreters)
- Experts/project creators analyze and (often) make publicly available the structured data. 

Several design attributes from Zooniverse are applicable to Dossier. In addition to high-quality handling of the associations between claim, warrant, and evidence (recording the chain of analysis, up to and including the version of the project that a warrant was built in), the design of the project encourages creation of structured and falsifiable information. ``The crowd'' is not an authoritative source for high-level analysis of scientific topics, but it is good at identifying well-defined entities, relationships, and even actions (Operation War Diary, for example https://www.operationwardiary.org/ structures information about troop movements from WWI military records.)

We can see another project built for managing claims extensibly in Salesforce. Even better, we can see its extension over the past decade, from 81 types of standard objects in Salesforce 7.0 (2006) to over 700 objects in version 40.0 (current). Given that the process of finding, nurturing, and closing a sales lead has not changed (tenfold) over the past eleven years, the salient point to be drawn from the development of Salesforce's ERD is the fact that preserving more and more subtle points drawn from real-world relationships, even within a fairly narrow field of human endeavor, means ever-increasing complexity. The market drives Salesforce's generalizability: Salesforce record is a nearly Borgesian attempt to map one type of relationship for as many possible companies as possible.


\section{To Develop}
The full paper will flesh out all of the above sections and also discuss
the following points

\begin{itemize}
  \item Allowing different journalists to attach resolve the same
    claims to different entities or accepting different claims as
    valid

    A DAG/tree model for using or interacting with the claims management system. Working in a scratch space to test out ``what if this John Doe is the same person as Jay Doe?''
  \item The parts of the problem that can be solved with off-the-shelf components.
  \item Maintaining fuzziness about categories, both to allow for changing understanding and flexibility in use of a system.
  \item Model claim dependencies. ``Person 1313515 gave over a hundred thousand dollars to candidates under a variety of names and companies'' depends on these fourteen other related claimes.
  \item The interface between the document management and claim
    management flows
  \item The interface of the dossier details views
  \item Allowing for the attachment of custom documents and extraneous
    information to a dossier
\end{itemize}











\end{document}
